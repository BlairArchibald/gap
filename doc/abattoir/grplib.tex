\Chapter{Group Libraries}

%%%%%%%%%%%%%%%%%%%%%%%%%%%%%%%%%%%%%%%%%%%%%%%%%%%%%%%%%%%%%%%%%%%%%%%%%
\Section{Basic groups}\null

\>PolyhedralGroup( \[ Is\dots Group \[ and \dots\ \] \], <p>, <q> )

returns the polyhedral group of size $<p> * <q>$. The numbers <p> and <q>
must be positive  integers and there must exist  a nontrivial <p>-th root
of unity modulo every prime factor of <q>.


%%%%%%%%%%%%%%%%%%%%%%%%%%%%%%%%%%%%%%%%%%%%%%%%%%%%%%%%%%%%%%%%%%%%%%%%%
\Section{Primitive permutation groups}

\)AllPrimitiveGroups( IsPrimitiveAffine, true, IsSolvable, true, %
                      $<fun>_3$, $<val>_3$, \dots\ )

If you use  the selection functions  for the primitive groups library and
select  only solvable affine  groups (as in  the example above), then you
can (in  the   place of the dots)    also give functions  that  you would
normally give   to `AllIrreducibleSolvableGroups'.  These functions  will
then be applied to the  point stabilizer of the affine  group, which is a
matrix group  from  M.~Short's library.  Hence you will  get only  groups
$V.M$  where  $V$ is  a    vector space and    $M$  a group  returned  by
`AllIrreducibleSolvableGroups( $<fun>_3$, $<val>_3$, \dots\ )'.

\>AllPrimitiveGroupsSims( $<fun>_1$, $<val>_1$, \dots\ )
\>OnePrimitiveGroupSims( $<fun>_1$, $<val>_1$, \dots\ )

For  compatibility with earlier versions  of {\GAP}, the original list of
Sims, with the same numbers and the names given by Buekenhout and Leemans
\cite{BuekenhoutLeemans96},  is also   included.  It is accessed  by  the
function  `PrimitiveGroupSims'. You can also  use the selection functions
with this old list, their names end with `Sims'.

%%%%%%%%%%%%%%%%%%%%%%%%%%%%%%%%%%%%%%%%%%%%%%%%%%%%%%%%%%%%%%%%%%%%%%%%%
\Section{Transitive permutation groups}

\>TransitiveGroup( <deg>, <nr> )

`TransitiveGroup'  returns the <nr>-th transitive  group of degree <deg>.
Both  <deg> and <nr> must be  positive integers. The transitive groups of
equal  degree are  sorted with  respect to   their  size, so for  example
`TransitiveGroup(  <deg>, 1 )' is a  transitive group  of degree and size
<deg>, e.g, the cyclic  group  of size <deg>,   if <deg> is a  prime. The
arrangement of  the groups, the  generators and their names correspond to
the lists in \cite{ConwayHulpkeMcKay97}

`AllTransitiveGroups'  and `OneTransitiveGroup'   recognize the following
functions and get the corresponding properties from a precomputed list to
speed up  processing:

`NrMovedPoints', `Size',   `Transitivity', and `IsPrimitive'.  You do not
need to pass  those functions first, as  the selection function picks the
these properties first.

The library was  brought into {\GAP} format by  Alexander  Hulpke, who is
responsible for all mistakes.

%%%%%%%%%%%%%%%%%%%%%%%%%%%%%%%%%%%%%%%%%%%%%%%%%%%%%%%%%%%%%%%%%%%%%%%%%
\Section{Finite perfect groups}

\>PerfectGroup( <selector> \[, <size> \[, <n> \] \] )

`PerfectGroup' is  the essential extraction function  of  the library. It
returns   a group, $G$   say, which is  isomorphic   to the library group
specified by   the  size number  [<size>,<n>]  or   by the  two  separate
arguments $size$ and  $n$,  assuming a default   value  of $n =  1$.  The
<selector> defines the representation,  in  which the group is  returned.
Possible selectors so far are `IsPermGroup' and `IsSubgroupFpGroup'

%%%%%%%%%%%%%%%%%%%%%%%%%%%%%%%%%%%%%%%%%%%%%%%%%%%%%%%%%%%%%%%%%%%%%%%%%%%%%
%%
%E

