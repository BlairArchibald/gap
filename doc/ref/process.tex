%%%%%%%%%%%%%%%%%%%%%%%%%%%%%%%%%%%%%%%%%%%%%%%%%%%%%%%%%%%%%%%%%%%%%%%%%
%%
%W  process.tex               GAP documentation              Frank Celler
%W                                                     & Martin Schoenert
%%
%H  @(#)$Id$
%%
%Y  Copyright 1997,  Lehrstuhl D fuer Mathematik,  RWTH Aachen,   Germany
%%
%%  This file contains the description of process.
%%


%%%%%%%%%%%%%%%%%%%%%%%%%%%%%%%%%%%%%%%%%%%%%%%%%%%%%%%%%%%%%%%%%%%%%%%%%
\PreliminaryChapter{Processes}

{\GAP}  can  call other programs,  such  programs are called *processes*.
There are  two  kinds of processes:  first there  are  processes that are
started, run and  return a result,  while  {\GAP} is suspended  until the
process  terminates.  Then there are  processes that will run in parallel
to {\GAP} as  subprocesses  and {\GAP} can  communicate and   control the
processes using streams.

Note that the latter kind of process is *@not yet implemented@*.

%%%%%%%%%%%%%%%%%%%%%%%%%%%%%%%%%%%%%%%%%%%%%%%%%%%%%%%%%%%%%%%%%%%%%%%%%
\Section{Process}

\>Process( <dir>, <prg>, <stream-in>, <stream-out>, <options> ) O

runs  a new process.  `Process' returns  when the process terminates.  It
returns the return value of the process (if the operating system supports
such a concept).

The first argument <dir> is a directory object which  will be the current
directory (in the usual UNIX  or MSDOS sense) when   the program is  run.
This will  only matter if  the program accesses  files (including running
other programs)  via relative path names.   In particular, it has nothing
to do with finding the binary to run.

In general the  directory will either  be the current directory, which is
returned by `DirectoryCurrent'  (which was the behaviour  of GAP 3), or a
temporary returned  by  `DirectoryTemporary'.  If  one  expects  that the
process creates temporary or log files the latter  should be used because
{\GAP} will attempt  to remove these  directories  together with all  the
files in them when quitting.

If a program of a share package which does not only consist of {GAP} code
needs  to be  launched  in a directory
relative  to  certain  data    libraries,   then  the first     entry  of
`DirectoryPackageLibrary'    should  be   used.     The     argument   of
`DirectoryPackageLibrary' should be the path to the data library relative
to the package directory.

If a program calls other programs and needs to be launched in a directory
containing the executables for such a share package then  the first entry of
`DirectoriesPackagePrograms' should be used.

The latter  two alternatives should only  be used if absolutely necessary
because otherwise one risks accumulating log or core files in the package
directory.

Examples

\beginexample
gap> path := DirectoriesSystemPrograms();;
gap> ls := Filename( path, "ls" );;
gap> stdin := InputTextUser();;
gap> stdout := OutputTextUser();;
gap> Process( path[1], ls, stdin, stdout, ["-c"] );;
awk    ls     mkdir

# current directory, here the root directory
gap> Process( DirectoryCurrent(), ls, stdin, stdout, ["-c"] );;
bin    lib    trans  tst    CVS    grp    prim   thr    two
src    dev    etc    tbl    doc    pkg    small  tom

# create a temporary directory
gap> tmpdir := DirectoryTemporary();;                          
gap> Process( tmpdir, ls, stdin, stdout, ["-c"] );;            
gap> PrintTo( Filename( tmpdir, "emil" ) );
gap> Process( tmpdir, ls, stdin, stdout, ["-c"] );;
emil
\endexample

<prg> is the filename of the program to launch, for portability it should
be   the  result  of   `Filename'   (see   "Filename") and   should  pass
`IsExecutableFile'.  Note that  `Process'  does *no* searching through  a
list of directories, this is done by `Filename'.

<stream-in>  is the  input stream   that  delivers the characters  to the
process.   For portability it  should either  be `InputTextNone' (if  the
process reads  no characters), `InputTextUser' (*@not yet implemented@*),
the  result  of a call to `InputTextFile'  from which  no characters have
been read, or the result of a call to `InputTextString'.

`Process' is  free to consume *all* the  input even if the program itself
does not require any input at all.

<stream-out> is the output stream  which receives the characters from the
process.  For portability it should   either be `OutputTextNone' (if  the
process writes no characters), `OutputTextUser' (*@not yet implemented@*),
the result of a call to `OutputTextFile' to which no characters have been
written, or the result of a call to `OutputTextString'.

<options> is a list of strings which are passed to the process as command
line argument.  Note that no substitutions are  performed on the strings,
i.e., they are passed immediately to the process and are not processed by
a command interpreter (shell).   Further note that  each string is passed
as one  argument,  even if it  contains  <space>  characters.  Note  that
input/output redirection commands are *not* allowed as <options>.

*Examples*

In   order to  find   a  system program  use  `DirectoriesSystemPrograms'
together with `Filename'.

\begintt
gap> path := DirectoriesSystemPrograms();;
gap> date := Filename( path, "date" );
"/bin/date"
\endtt

Now execute `date' with no argument and no input, collect the output into
a string stream.

\begintt
gap> str := "";; a := OutputTextString(str,true);;
gap> Process( DirectoryCurrent(), date, InputTextNone(), a, [] );
0
gap> CloseStream(a);
gap> Print(str);   
Fri Jul 11 09:04:23 MET DST 1997
\endtt

%%%%%%%%%%%%%%%%%%%%%%%%%%%%%%%%%%%%%%%%%%%%%%%%%%%%%%%%%%%%%%%%%%%%%%%%%
\Section{Exec}

\>Exec( <cmd>, <option1>, ..., <optionN> ) F

runs a shell in the  current directory to execute  the command <cmd> with
options <option1> ... <optionN>.

\begintt
gap> Exec( "date" )
Thu Jul 24 10:04:13 BST 1997
\endtt

<cmd> is  interpreted by the shell  and therefore we  can make use of the
various features that a shell offers as in following example.

\beginexample
gap> Exec( "echo \"GAP is great!\" > foo" );    
gap> Exec( "cat foo" );
GAP is great!
gap> Exec( "rm foo" );
\endexample

%%%%%%%%%%%%%%%%%%%%%%%%%%%%%%%%%%%%%%%%%%%%%%%%%%%%%%%%%%%%%%%%%%%%%%%%%
%%
%E  streams.tex . . . . . . . . . . . . . . . . . . . . . . . . ends here
