%%%%%%%%%%%%%%%%%%%%%%%%%%%%%%%%%%%%%%%%%%%%%%%%%%%%%%%%%%%%%%%%%%%%%%%%%%%%
%
%A help.tex            GAP documentation                Martin Schoenert
%
%A @(#)$Id$
%
%Y Copyright 1990-1992, Lehrstuhl D fuer Mathematik, RWTH Aachen, Germany
%
\Chapter{The Help System}

This chapter describes the {\GAP} help system.
The help system lets you read the manual interactively.

%%%%%%%%%%%%%%%%%%%%%%%%%%%%%%%%%%%%%%%%%%%%%%%%%%%%%%%%%%%%%%%%%%%%%%%%
\Section{Help}

\>`?<section>'{getting help}

The help command `?' displays the section with the name <section> on the
screen. For example `?Help' will display this section on the screen.
You should not type in the single quotes, they are only used in help
sections to delimit text that you should enter into {\GAP} or that {\GAP}
prints in response. When the whole section has been displayed the normal
{\GAP} prompt `gap>' is shown and normal {\GAP} interaction resumes.

When there are several manual sections that match `section' a list of topics
as described under "Help Index" is displayed.

Section~"Reading Sections" tells you what actions you can perform
while you are reading a section. You tell {\GAP} to display this
section by entering `?Reading Sections', without quotes.
Section~"Format of Sections" describes the format of sections and the
conventions used,
"Browsing through the Sections" lists the commands you use to flip
through sections, "Redisplaying a Section" describes how to read a
section again, "Abbreviating Section Names" tells you how to avoid typing
the long section names, and "Help Index" describes the index command.

%%%%%%%%%%%%%%%%%%%%%%%%%%%%%%%%%%%%%%%%%%%%%%%%%%%%%%%%%%%%%%%%%%%%%%%%
\Section{Reading Sections}

\index{help!scrolling}
If the section is longer than the size of your screen {\GAP} stops after
displaying a full screen and displays

\begintt
-- <space> for more, <q> to quit --
\endtt

If you press <space> {\GAP} displays the next 24 lines of the section and
then stops again. This goes on until the whole section has been
displayed, at which point {\GAP} will return immediately to the main
{\GAP} loop. Pressing `f' has the same effect as <space>.

You can also press `b' which will scroll back to
the *previous* 24 lines of the section. If you press `b' when
{\GAP} is displaying the top of a section {\GAP} will ring the bell.

You can also press `q' to quit and return immediately back to the main
{\GAP} loop without reading the rest of the section.

The size of the screen is set by default to 24 lines.  If you have a larger
screen that can display more lines of text you may want to tell this to
{\GAP} with the `-y <rows>' option when you start {\GAP}.

%%%%%%%%%%%%%%%%%%%%%%%%%%%%%%%%%%%%%%%%%%%%%%%%%%%%%%%%%%%%%%%%%%%%%%%%
\Section{Format of Sections}

\index{help!format}
This section describes the format of sections when they are displayed on
the screen and the special conventions used.
For general conventions about manual sections and the format of sections
in the printed manual, see~"Manual Conventions".

As you can see, {\GAP} prints a header line
containing the name of the section on the left and the name of the
chapter on the right.

(If this header line ends in `(not loaded)' the documentation belongs to a
share package, which has not yet been loaded. See section~"Loading a Share
Package" for further information.)

\begintt
<text>
\endtt
Text enclosed in angle brackets is used for arguments in the descriptions
of functions and for other place holders. It means that you should not
actually enter this text into {\GAP} but replace it by an appropriate
text depending on what you want to do. For example when we write that
you should enter `?<section>' to see the section with the name <section>,
<section> serves as a place holder, indicating that you can enter the
name of the section that you want to see at this place.

\begintt
`text'
\endtt
Text enclosed in single quotes is used for names of variables and
functions and other text that you may actually enter into your computer
and see on your screen. The text enclosed in single quotes may contain
place holders enclosed in angle brackets as described above. For example
when the help text for `IsPrime' says that the form of the call is
`IsPrime( <n> )' this means that you should actually
enter the strings ``IsPrime('' and ``)'', without the quotes,
but replace the `<n>' with the number (or expression)
that you want to test.

\begintt
"text"
\endtt
Text enclosed in double quotes is used for cross references to other
parts of the manual. So the text inside the double quotes is the name of
another section of the manual. This is used to direct you to other
sections that describe a topic or a function used in this section. So
for example "Browsing through the Sections" is a cross reference to the next
section.

\begintt
> Oper( <arg1>, <arg2>[, <opt>] ) F
\endtt
starts a subsection on the command `Oper' that gets two arguments <arg1>
and <arg2> and an optional third argument <opt>.

The letter `F' at the end
indicates that the command is a simple function.
The letters `A', `P', `O', `C', `R', and `V' indicate
``Attribute'', ``Property'', ``Operation'', ``Category'', ``Representation''
(see Chapter~"Types of Objects"), or ``Variable'', respectively.

`_' and `^'

In mathematical formulas the underscore and the caret are used to denote
subscription and superscription. Ordinarily they apply only to the very
next character following, unless a whole expression enclosed in
parentheses follows. So for example `x_1^(i+1)' denotes the variable `x'
with subscript 1 raised to the `i+1' power.

Longer examples are usually paragraphs of their own.
Everything on the lines with the prompts `gap>' and `>', except
the prompts themselves of course, is the input you have to type,
everything else is {\GAP}'s response.

\begintt
gap> ?Format of Sections
Format of Sections ______________________________________ Environment

This section describes the format of sections when they are displayed
on the screen and the special conventions used.
... 
\endtt

%%%%%%%%%%%%%%%%%%%%%%%%%%%%%%%%%%%%%%%%%%%%%%%%%%%%%%%%%%%%%%%%%%%%%%%%
\Section{Browsing through the Sections}

\index{help!browsing}
The help sections are organized like a book into chapters. This should
not surprise you, since the same source is used both for the printed
manual and the online help. Just as you can flip through the pages of a
book there are special commands to browse through the help sections.

\>`?>'{browsing forward}
\>`?\<'{browsing backwards}

The two help commands `?\<' and `?>' correspond to the flipping of pages.
`?\<' takes you to the section preceding the current section and displays
it, and `?>' takes you to the section following the current section.

\>`?>>'{browsing forward one chapter}
\>`?\<\<'{browsing backwards one chapter}

`?\<\<' is like `?\<', only more so. It takes you back to the first
section of the current chapter, which gives an overview of the sections
described in this chapter. If you are already in this section `?\<\<'
takes you to the first section of the previous chapter. `?>>' takes you
to the first section of the next chapter.

\>`?-'{browsing the previous section browsed}
\>`?+'{browsing the next section browsed}

{\GAP} remembers the sections that you have read. `?-' takes you to the
one that you have read before the current one, and displays it again.
Further `?-' takes you further back in this history. `?+' reverses this
process, i.e., it takes you back to the section that you have read
*after* the current one. It is important to note, that `?-' and `?+' do
*not* alter the history like the other help commands.


%%%%%%%%%%%%%%%%%%%%%%%%%%%%%%%%%%%%%%%%%%%%%%%%%%%%%%%%%%%%%%%%%%%%%%%%
\Section{Changing the Way the Help Pages are Displayed}

\index{HTML}
\index{netscape}\index{lynx}\index{internet config}
\index{less}\index{pager}

If you have installed an html version of the manual you can
alternatively use an html browser to display the manual sections. The
command

\>SetHelpViewer(<device>)

will select a method described by the string <device> to display the help
pages online. Currently <device> can be `"screen"' for the default built-in
text browser, `"less"' (only under UNIX) to use `less' for paging,
`"netscape"' for the netscape HTML browser and `"lynx"' for the
lynx HTML browser. Both HTML browsers will only work under UNIX. 

On an Apple Macintosh you can use an HTML browser by calling `SetHelpViewer'
with the parameter `"Internet Config"'.
See section~"Features of GAP for MacOS" for details about this.

If you want the HTML help to be the default, you should call this function
in your `.gaprc' file (see the sections on operating
system dependent features in chapter "Installing GAP").

%%%%%%%%%%%%%%%%%%%%%%%%%%%%%%%%%%%%%%%%%%%%%%%%%%%%%%%%%%%%%%%%%%%%%%%%
\Section{Redisplaying a Section}

\index{help!redisplaying}

\>`?'{browsing the same section again}

The help command `?' followed by no section name redisplays the last help
section again. So if you reach the bottom of a long help section and have
already forgotten what was mentioned at the beginning, or, for example, the
examples do not seem to agree with your interpretation of the
explanations, use `?' to read the whole section again from the beginning.

When `?' is used before any section has been read {\GAP} displays a
`Welcome to GAP'.

%%%%%%%%%%%%%%%%%%%%%%%%%%%%%%%%%%%%%%%%%%%%%%%%%%%%%%%%%%%%%%%%%%%%%%%%
\Section{Abbreviating Section Names}

\index{help!abbreviating}
Upper and lower case in <section> are not distinguished, so typing either
`?Abbreviating Section Names' or `?abbreviating section names' will show
the section you are currently reading.

Each word in <section> may be abbreviated. So instead of typing
`?abbreviating section names' you may also type `?abb sec nam', or even `?a
s n'. You must not omit the spaces separating the words. For each word in
the section name you must give at least the first character. As another
example you may type `?el oper for int' instead of `?elementary operations
for integers', which is especially handy when you can not remember whether
it was `operations' or `operators'.

If an abbreviation matches multiple section names a list of all these
section names is displayed.

%%%%%%%%%%%%%%%%%%%%%%%%%%%%%%%%%%%%%%%%%%%%%%%%%%%%%%%%%%%%%%%%%%%%%%%%
\Section{Help Index}

\>`??<topic>'{list help topics}

The operator `??' looks up <topic> in {\GAP}'s index and prints all the
index entries that contain the substring <topic>.
Then you can decide which section is the one you are actually interested
in and request this one.

\begintt
gap> ??read
Help: several entries match this topic
[1] reference:read
[2] reference:reading sections
[3] reference:readlib
[4] reference:isreadablefile
[5] reference:read
...
\endtt

The first thing on each line is a reference number.
Then follows the part of the manual which contains the section and
finally the actual name of the (sub)section. All names are converted to
lower case.

The order of the sections corresponds to their order in the
{\GAP} manual, so that related sections should be adjacent.

You can then either refer to the desired subsection by their name or simply
use `?<nr>' to look at the topic with the reference number <nr>. So in the
above example `?3' would display the section on `ReadLib'.

When referring to sections by their name you can usually omit the part
of the manual unless several parts contain the same section names.

If there are several subsections which have exactly the same name a number
in parentheses is added to the name to distinguish these.

%%%%%%%%%%%%%%%%%%%%%%%%%%%%%%%%%%%%%%%%%%%%%%%%%%%%%%%%%%%%%%%%%%%%%%%%%
%%
%E

