%%%%%%%%%%%%%%%%%%%%%%%%%%%%%%%%%%%%%%%%%%%%%%%%%%%%%%%%%%%%%%%%%%%%%%%%%
%%
%W  preface.tex               GAP documentation         Joachim Neubueser
%%
%H  @(#)$Id$
%%
%Y  Copyright (C) 1997, Lehrstuhl D fuer Mathematik, RWTH Aachen, Germany
%%
%%  This file contains the preface of the GAP manual.
%%
%%%%%%%%%%%%%%%%%%%%%%%%%%%%%%%%%%%%%%%%%%%%%%%%%%%%%%%%%%%%%%%%%%%%%%%%%
\Chapter{Preface}

{\GAP} stands  for  *Groups, Algorithms  and Programming*.  The  name was
chosen to reflect  the aim of  the  system, which  is introduced in  this
tutorial manual.  Since  that  choice,  the  system has become   somewhat
broader,    and  you will  also   find information   about algorithms and
programming  for other   algebraic structures,   such as semigroups   and
algebras.

There are three further manuals in addition to  this one: the ``Reference
Manual''    containing   detailed  documentation     of  the mathematical
functionality   of {\GAP}; ``Extending   GAP''  containing  some tutorial
material on various aspects  of {\GAP} programming; and  ``Programming in
GAP 4''   containing detailed documentation   of various aspects   of the
system of interest mainly to programmers.

{\GAP}  is  a  *free*, *open*  and    *extensible*  software package  for
computation  in  discrete abstract  algebra.    The terms ``free''  and
``open'' describe the conditions under which the system is distributed --
in brief, it is *free of charge* (except possibly for the immediate costs
of delivering it to  you), you are *free  to pass  it on* within  certain
limits,  and all  of the  workings of  the  system are  *open  for you to
examine and   change*. Details of these  conditions  can be  found in the
reference     manual in chapter "ref:Copyright  Notice".    The system is
``extensible''  in that  you can write  your  own programs in the  {\GAP}
language, and use them  in just the same way  as the programs  which form
part of  the system  (the  ``library''). Indeed,  we actively support the
contribution, refereeing and distribution of extensions to the system, in
the form of ``share packages''.  Further details of  this can be found in
chapter  "ref:Share Packages" in the  reference manual,  and on our World
Wide Web site.

Development of {\GAP} began at Lehrstuhl D f\accent127ur Mathematik, RWTH-Aachen,
under the leadership of Prof.~Joachim Neub\accent127user in 1985. Version 2.4 was
released in 1988 and   version 3.1 in 1992.  The  final full release   of
{\GAP} 3, version 3.4, was  made in 1994. A more  detailed account of the
motivation and development  of these versions  of {\GAP} is contained  in
section "From the Preface for GAP 3.4, June 1994", below.

Since then,  there  have  been  two  dramatic transitions in  the  {\GAP}
project.  Firstly,    in  1997, Prof.~Neub\accent127user   retired,   and overall
coordination of {\GAP}  development,   now  very much   an  international
effort, was transferred  to  St Andrews.   Secondly  a complete  internal
redesign and almost complete rewrite of the  system, which was already in
progress in Aachen,  has  been completed.  Following  five, increasingly
usable,   beta-test releases, version  4.1 is  the  first  version of the
rewritten system to be released without qualification for general use. 

For those readers who have used an earlier version of {\GAP}, an overview
of the changes from the beta-test releases and from {\GAP} 3 is given in
section "Changes from Earlier Versions" below.

The system that you are getting now consists of four parts.
\beginlist
  \item{1.}
    A  *kernel*, written in C, which provides the user with
    \itemitem{-}
      automatic dynamic storage management, which the user needn't bother
      about in his programming;
    \itemitem{-}
      a   set of  time-critical basic   functions, e.g.   ``arithmetic'',
      operations for integers, finite fields,  permutations and words, as
      well as natural operations for lists and records;
    \itemitem{-} 
      an interpreter   for  the {\GAP} language,    an untyped
      imperative programming language with functions as first class objects
      and some extra built-in data types such as permutations and finite
      field elements. 
    \itemitem{-}
      support for the time critical parts of the new {\GAP}4 method
      selection system.
    \itemitem{-}
      a small set of system functions allowing the {\GAP} programmer to handle
      files and execute external programs in a uniform way, regardless of 
      the particular operating system in use. 
    \itemitem{-}
      a   set  of programming tools  for   testing, debugging, and timing
      algorithms.

  \item{2.}
    A   much larger *library of    {\GAP} functions* that implement algebraic
    and other algorithms.  Since  this is written entirely in
    the  {\GAP} language, the  {\GAP}    language   is both   the  main
    implementation language   and  the  user  language of   the   system.
    Therefore  the  user can    as easily  as  the  original  programmers
    investigate  and vary algorithms of  the library and  add new ones to
    it, first for own  use and eventually for  the benefit of  all {\GAP}
    users.  

  \item{3.}  
    A *library of group theoretical data* which already contains
    various libraries of groups, large libraries of ordinary and
    Brauer  character tables,   including  all of the   Cambridge *Atlas of
    Finite Groups*  and *Atlas of Brauer Characters*,  a *library of tables
    of marks*, a *library of small groups* (containing  all groups of order
    at most 1000, except those of order 512) and others.


  \item{4.}
    The *documentation*.  This is available as files that can either be
    used for on-line  help or processed for printing with \TeX\ or into
    HTML for viewing with a browser such as `netscape'.
\endlist

Together  with the system we distribute  *GAP  share packages*, which are
separate packages which have been written by various groups of people and
remain  under their responsibility.  Some  of  these packages are written
completely in the {\GAP}  language, others totally   or in part in  C (or
even  other languages).  However the functions   in these packages can be
called directly from {\GAP} and results are returned to {\GAP}.

We  operate a refereeing  system  for such packages,  both  to ensure the
quality of the software we distribute, and to provide recognition for the
authors.    A  number of  {\GAP}   4  share   packages  have already been
accepted. Some others  that have been submitted  for review are available
as  ``preprints''. More information  can be found on   our World Wide Web
site, see section "Further Information about GAP".


In the preface to the first beta release of {\GAP} 4, Joachim Neub\accent127user wrote:

\begingroup\advance\leftskip 0.5in\advance\rightskip 0.5in

   $\ldots$  It remains to me to  thank all those  who have done the huge
   amount of work that  was needed to  bring  {\GAP}~4 on its  way.  Many
   basic ideas for  the new concepts as  well  as most of the  new kernel
   implementation are still due  to Martin Sch\accent127onert  before and
   even  in parts  after  he left  Lehrstuhl D  f\accent127ur Mathematik.
   However   together with  him  while   he  was still  working  here and
   continuing after he left, Thomas Breuer and  Frank Celler have in long
   discussions found  the way to  the concepts and  done crucial parts of
   the  new  implementations.   Many  others  have  worked  adapting  and
   rewriting the library, of whom I want to mention in particular Bettina
   Eick, Alexander  Hulpke and Heiko  Theissen  from the Aachen  team but
   also   acknowledge   the help  lent   already  for  some  time  from
   St.~Andrews, in particular by Steve Linton.

   To these and all others, whom I did not mention  explicitly, I want to
   express    my thanks for   a  yearlong  cooperation    in a  spirit of
   enthusiasm,   dedication  and perseverance.    I   wish  the team   at
   St.~Andrews  a   successful   continuation   of  the   development and
   maintenance   of {\GAP}  in  that same  spirit and  all  users fun and
   success in using {\GAP}.

   Aachen, July 18, 1997 \hfill Joachim Neub\accent127user
\endgroup

Before going  on to mention more  recent contributions, I must express my
heartfelt thanks, and I am  sure, that of  the wider community of  {\GAP}
developers and users for the immense work that Joachim Neub\accent127user
himself has put into {\GAP} over many years.  The system would never have
existed, let  alone  grown and prospered  as it   has, without his  clear
vision of  what  he wanted  it  to  become, his ceaseless  vigilance  for
opportunities    for development, his    championing  of   the  cause  of
computation  in group theory,  his high standards   which would not admit
``merely adequate'' solutions and  his constant encouragement of everyone
working on and with the system. We are all the richer for his efforts.


Many people  have contributed  to {\GAP}  development over the  last  two
years.  Alexander  Hulpke has been  a tower of strength, coordinating the
work on  the library, finding subtle bugs  in code whose authors  were no
longer  contactable,  supporting the  beta   testers and,  most  recently
coordinating the process of   stabilization and debugging leading  up  to
this release. Thomas Breuer   in Aachen has remained our  ``conscience'',
pointing out to us when we were misusing {\GAP} 4 concepts and so storing
up trouble for later. He has also done an  enormous amount of work on the
support for  representation theory in   {\GAP}.  Others I  would like  to
point out here  include Bettina Eick,  Volkmar  Felsch, Willem de  Graaf,
Werner  Nickel and Andrew  Solomon.  I should like  to thank all of these
people,  and   all the other   contributors  whom  I  have not  mentioned
explicitly for their efforts, their  cheerfulness and their perseverance.
I  should also   like  to  thank  all the   {\GAP}4 beta-testers, package
developers,  manual proof readers and others  who have given us extremely
valuable and positive feedback.

{\GAP} development in  St Andrews has  been financially supported  by the
Engineering and Physical Sciences Research Council, the Leverhulme Trust,
the European   Commission  (ESPRIT  programme),  the  Royal   Society  of
Edinburgh and the  British Council, to all of  whom we are very grateful.
Development  also takes  place at other  centres  with support from other
funding bodies.

It finally remains  for me to wish you  all pleasure and success in using
{\GAP}, and to invite your constructive comment and criticism.

St Andrews 26 July 1999\hfill Steve Linton


%%%%%%%%%%%%%%%%%%%%%%%%%%%%%%%%%%%%%%%%%%%%%%%%%%%%%%%%%%%%%%%%%%%%%%%%%
\Section{Changes from Earlier Versions}

The main changes from the final beta-test release {\GAP}  4 beta 5 are in
the documentation and in performance, both of which are much improved and
in the   fixing of many  bugs. The  installation  process  has  also been
improved a  little, and there are  some new algorithms, especially in the
areas of semigroups and finitely-presented groups.  As far as we know any
programs that worked with 4 beta 5 should still work in {\GAP} 4.1.

The changes since the final release of  {\GAP} 3 (version 3.4.4) are much
more wide-ranging.   The general philosophy  of  the changes in two-fold.
Firstly, many assumptions in the design of {\GAP} 3 revealed its authors'
primary  interest    in  group  theory,  and  indeed     in  finite group
theory. Although much of the  {\GAP} 4 library  is concerned with groups,
the basic  design now allows extension  to other algebraic structures, as
witnessed  by   the inclusion of   substantial  bodies of  algorithms for
computation with semigroups and Lie  algebras.  Secondly, as the scale of
the  system, and the  number of people  using and contributing  to it has
grown, some   aspects of  the   underlying   system have  proved   to  be
restricting,  and   these have  been  improved as  part  of comprehensive
re-engineering of the system. This has  included the new method selection
system, which underpins the library, and a new, much more flexible, share
package interface.

Details of these changes can be found in chapter "Migrating to GAP 4" of
this manual. It is perhaps worth mentioning a few points here. 

Firstly, much remains unchanged, from the perspective of the mathematical 
user:

\beginlist
  \item{-}
    The syntax of that part  of the {\GAP} language  that most users need
    for investigating mathematical problems.

  \item{-}
    The great majority of function names.

  \item{-}
    Data libraries and the access to them.
\enditems

A number of visible aspects have changed:

\beginlist
  \item{-}
    Some function names that need finer specifications now that there are
    more structures available in {\GAP}.

  \item{-}  
    The    access to information   already  obtained about a mathematical
    structure. In {\GAP}~3 such information about a group could be looked
    up  by  directly inspecting  the  group record,  whereas in  {\GAP} 4
    functions must be used to access such information.
\endlist

Behind the scenes, much has changed: 

\beginlist
  \item{-} A new kernel,  with improvements in  memory management  and in
  the language interpreter, as well  as new  features  such as saving  of
  workspaces and the possibility of compilation of {\GAP} code into C.

  \item{-} A new structure   to the library, based  upon  a new  type and
  method  selection system, which  is able to  support a broader range of
  algebraic computation and to make the  structure of the library simpler
  and more modular.

  \item{-}
    New and faster algorithms in many mathematical areas.

  \item{-} 
    Data structures and algorithms for new mathematical objects, such as
    algebras and semigroups.

  \item{-} A new and more flexible structure  for the {\GAP} installation
  and documentation, which  means, for example, that  a share package and
  its documentation can be installed and be fully usable without any changes
  to the {\GAP} system.
\endlist

A very few features of {\GAP} 3 are not yet available in  {\GAP} 4.

\beginlist
   \item{-}
     Only a few of the {\GAP} 3 share packages have yet been converted
     for use with  {\GAP} 4 (although several new packages are available
     only in  {\GAP} 4). 

   \item{-} The Galois group determination algorithms which were
      implemented in the {\GAP} 3 library are not present in {\GAP} 4.
    
   \item{-} The algorithms for the factorization of polynomials over
   algebraic number fields which were implemented in the {\GAP} 3 library
   are not present in {\GAP} 4.

   \item{-} The library of crystallographic groups which was present in
     {\GAP} 3 is now part of a share package `crystcat', which has been
     submitted for refereeing and is distributed with this release as a
     ``preprint''.

   \item{-}  The library  of  irreducible maximal finite  integral matrix
   groups is not yet available.

   \item{-} The p-quotient and soluble quotient algorithms are
   implemented in the {\GAP} 4 library, but those implementations are not
   yet described in the documentation. If you have a pressing need to use
   them, please contact `gap-trouble'.
\endlist
      






%%%%%%%%%%%%%%%%%%%%%%%%%%%%%%%%%%%%%%%%%%%%%%%%%%%%%%%%%%%%%%%%%%%%%%%%%
\Section{From the Preface for GAP 3.4, June 1994}

{\GAP} stands for *Groups,  Algorithms  and  Programming*.  The name  was
chosen to reflect the  aim of the  system,  which is  introduced in  this
manual.

Until  well into the  eighties  the  interest  of pure mathematicians  in
computational  group theory  was  stirred  by,  but in  most  cases  also
confined to  the  information  that  was  produced  by  group theoretical
software  for  their special research  problems  --  and  hampered by the
uneasy  feeling  that  one  was   using  black  boxes  of  uncontrollable
reliability.  However the last years have seen a rapid spread of interest
in the understanding, design and even implementation of group theoretical
algorithms.  These are gradually becoming accepted both as standard tools
for a working group theoretician,  like certain  methods of proof, and as
worthwhile  objects of study, like  connections between notions expressed
in theorems.

{\GAP} was  started as  an attempt to meet  this  interest.   Therefore a
primary design goal has  been to give its user full access  to algorithms
and the data  structures used  by them, thus  allowing  critical study as
well as  modification of existing methods.  We also intend to relieve the
user from unwanted technical chores and to assist him in the programming,
thus supporting invention and implementation of new algorithms as well as
experimentation with them.

We have tried  to achieve these goals by a design which in addition makes
{\GAP} easily portable, even to computers such as Atari ST and Amiga, and
at the same  time facilitates the maintenance of {\GAP} with  the limited
resources of an academic environment.

While I had felt for some time rather strongly  the wish for such a truly
*open* system for computational group theory, the concrete idea of {\GAP}
was born when, together with a larger group of  students, among whom were
Johannes   Meier,    Werner   Nickel,   Alice     Niemeyer,   and  Martin
Sch\accent127onert who eventually wrote the first  version of {\GAP}, I
had my first contact   with the Maple system   at the EUROCAL  meeting in
Linz/Austria  in  1985.  Maple demonstrated   to us the feasibility  of a
strong  and efficient computer algebra system  built from a small kernel,
with an  interpreted library of   routines written in  a  problem-adapted
language.  The discussion of the plan of a system for computational group
theory organized  in    a similar  way  started  in  the  fall  of  1985,
programming only in the second half  of 1986.  A  first version of {\GAP}
was operational by  the end of 1986.  The  system was first  presented at
the Oberwolfach meeting    on computational group   theory  in May  1988.
Version  2.4  was  the first  officially  to  be  given  away from Aachen
starting in December 1988.  The strong interest in this version, in spite
of its  still rather small  collection of group theoretical  routines, as
well  as constructive criticism  by many colleagues, confirmed our belief
in the general design principles of the  system.  Nevertheless over three
years had passed until in April 1992  version 3.1 was released, which was
followed in February 1993 by version 3.2, in November 1993 by version 3.3
and is now in June 1994 followed by version 3.4.

A main reason for the long time between versions 2.4 and 3.1 and the fact
that there had not been  intermediate releases was that  we had found  it
advisable to make a number of changes to basic data structures until with
version 3.1 we  hoped  to have reached a   state where we could  maintain
upward compatibility over further  releases, which were planned to follow
much more frequently.  Both  goals have been  achieved over the last  two
years. Of course the time has  also been used to extend  the scope of the
methods implemented in {\GAP}.   A rough estimate   puts the size  of the
program library of version 3.4 at about sixteen times the size of that of
version 2.4, while for version 3.1 the factor  was about eight.  Compared
to {\GAP}~3.2,  which  was the  last version  with  major  additions, new
features of {\GAP}~3.4 include the following:

$\ldots$

{\GAP} was started as a joint Diplom project of four students whose names
have    already been mentioned.  Since   then   many more finished Diplom
projects  have   contributed to {\GAP}   as  well   as other  members  of
Lehrstuhl~D and  colleagues   from other institutes.    Their  individual
contributions  to the  programs and to  the manual  are documented in the
respective  files.  To all  of  them as  well as to   all who have helped
proofreading and  improving this manual  I want to  express my thanks for
their engagement and  enthusiasm as well as  to many users of {\GAP}  who
have helped us by pointing  out deficiencies and suggesting improvements.
Very special thanks however go   to Martin Sch\accent127onert.  Not  only
does   {\GAP}  owe many of its   basic  design  features  to his profound
knowledge of     computer   languages  and   the   techniques   for their
implementation, but in many long discussions he has in the name of future
users always been the strongest defender of clarity of the design against
my impatience and the temptation for ``quick and dirty'', solutions.

Since  1992 the development of  {\GAP}  has been financially supported by
the Deutsche     Forschungsgemeinschaft    in  the   context      of  the
Forschungsschwerpunkt  ``Algorithmische Zahlentheorie   und  Algebra''.
This very important help is gratefully acknowledged.

As with the previous versions we send this version out hoping for further
feedback of constructive   criticism.  Of course  we ask  to be  notified
about bugs,  but moreover  we shall  appreciate   any suggestion  for the
improvement of the  basic  system as  well  as of  the algorithms  in the
library.  Most of all,  however, we hope that in  spite of such criticism
you will enjoy working with {\GAP}.

Aachen, June 1.,1994, \hfill Joachim Neub\accent127user.

%%%%%%%%%%%%%%%%%%%%%%%%%%%%%%%%%%%%%%%%%%%%%%%%%%%%%%%%%%%%%%%%%%%%%%%%%%
%\Section{Preface for the first beta release of GAP 4}

%The transition from {\GAP}~3.4 which got its presumably last update 3.4.4
%in April  this year to this  present first beta  release {\GAP}~4.B.1  of
%{\GAP}~4  marks a  major step in the system  design of {\GAP}, similar in
%importance to the  step from {\GAP}~2.4 to   {\GAP}~3.1 in April 1992  on
%which I comment in my preface  to {\GAP}~3.4 of  June 1994, most of which
%is preceding this  preface. However in  contrast to the situation in 1992
%we hope that the  changes will be much less  bothering to the majority of
%the  {\GAP} users this time.   Let me first talk   about some reasons and
%background  for developing {\GAP}~4  and then briefly sketch what remains
%and what changes.

%The  planning of {\GAP}~4 started  already at the time  of the release of
%{\GAP}~3.4 (Summer 1994) and its development has been  a major reason for
%the fact that  since then  only updates  (up to {\GAP}~3.4.4)  but no new
%releases of {\GAP}~3 have come out. Also a number  of new algorithms have
%been implemented   in  Aachen anticipating {\GAP}~4   and  hence have not
%become generally available yet.

%There were three major reasons for the development of {\GAP}~4:

%\beginlist
%  \item{-}
%    There has  been a  growing   demand  to implement  new   mathematical
%    structures in  {\GAP} (Lie algebras are just  one  example).  However
%    {\GAP}~3 was not really designed for such tasks.

%  \item{-}
%    The number and diversity of (sometimes competing) algorithmic methods
%    is  growing rapidly.   We  definitely want  to maintain the principle
%    that the  user  should  be  able to control   what  methods are used.
%    However, the growing complexity  of  the interrelation of  algorithms
%    makes it mandatory to have  also a `method selection' mechanism which
%    controls the choice between different possibilities to proceed within
%    a  {\GAP} function,  that  is  at least  partially  guided by already
%    computed knowledge about the objects under investigation.

%  \item{-}
%    While the two  points mentioned above  have  caused `visible' changes
%    from {\GAP}~3 to {\GAP}~4, in  this transition also important changes
%    have taken place `behind the scene'.  There are e.g.  improvements of
%    the  storage  management and function  calls,   and last  not least a
%    compiler from {\GAP}~4 to C is part of this beta release.
%\endlist

%So regarding system aspects let's briefly sketch:

%What is left unchanged?

%\beginlist
%  \item{-}
%    The syntax of that part  of the {\GAP} language  that most users need
%    investigating mathematical problems.

%  \item{-}
%    The great majority of function names.

%  \item{-}
%    Data libraries and the access to them.
%\enditems

%What has changed? 

%\beginlist
%  \item{-}
%    Some function names that need finer specifications now that there are
%    more structures available in {\GAP}.

%  \item{-}
%    The access to  information   already obtained about  a   mathematical
%    structure.  E.g.  in {\GAP}~3 such information about a group could be
%    looked up by directly inspecting the  group record, whereas in {\GAP}
%    4 functions must be used to access such information.
%\endlist

%What is new?

%\beginlist
%  \item{-}
%    A whole machinery for the definition of new structures. 

%  \item{-}
%    A  hopefully  clearer separation of  aspects  of knowledge  about the
%    mathematical objects that {\GAP} handles  by the introduction of  the
%    concepts of attributes, families, categories, and representations.

%  \item{-}
%    A number of new structures, such as  Lie algebras.
%\enditems

%Then to the mathematical functionality provided by {\GAP}~4 in comparison
%to {\GAP}~3:

%\beginlist
%  \item{-}
%    Almost all of the program library of {\GAP}~3 has been transferred to
%    {\GAP}~4.   Some of this had  just to be adapted to  the new features
%    (which in  itself has been  a huge  job in view   of the size  of the
%    program library), but for quite a  few tasks the opportunity has been
%    used to implement new and more efficient algorithms - notably so e.g.
%    for permutation groups and polycyclic groups.

%  \item{-}
%    A number    of new algorithms  have  been  implemented for  which the
%    features of {\GAP}~4 proved more adequate or even necessary and which
%    are now made public in {\GAP} for the first time.

%  \item{-}
%    One main deficiency of the present beta release is that the meanwhile
%    large   library of share  packages  of   {\GAP}~3  has not  yet  been
%    transferred.
%\endlist

%The other main deficiency is that there is not yet  a complete manual for
%{\GAP}~4.

%It is intended to provide eventually at least four  books as parts of the
%manual. The  first and second  are intended for  people  who want  to use
%{\GAP} ``as is''.  Books 3 and 4  on the other hand  are meant for people
%who want to extend {\GAP}~4 by introducing new structures.  Books 1 and 3
%are  tutorials for the respective   purpose while Books  2 and  4 are the
%corresponding reference manuals.

%Of these four books a  good deal  of Books 1   and 3, i.e.  the  tutorial
%parts, are provided with  this release, while  there are only rudimentary
%parts  of books  2  and 4  available.   For people  already familiar with
%{\GAP}~3 the chapter ``Migrating  to {\GAP}~4'' in the  first book may be
%particularly helpful.

%There have been (weekly  changing) alpha test versions  of {\GAP}~4 since
%December  1996, and a number   of specially experienced  {\GAP} users  in
%addition to  the {\GAP} teams at Aachen  and St.~Andrews  have used these
%and   provided helpful criticism and  suggestions.   It is envisaged that
%there will be new beta releases from now on about  every couple of months
%until an  official version  {\GAP}~4.1 can be released  next year.  It is
%hoped that these further beta releases will gradually provide the missing
%parts mentioned above as well as further enhancements. However, since the
%date of the official  handover of {\GAP} from  Aachen to  St.~Andrews has
%now been fixed to be July 21, 1997, this  further development will happen
%under the responsibility of St.~Andrews.


%%%%%%%%%%%%%%%%%%%%%%%%%%%%%%%%%%%%%%%%%%%%%%%%%%%%%%%%%%%%%%%%%%%%%%%%%
\Section{Further Information about GAP}


Information about {\GAP} is best obtained from the {\GAP} Web pages that
you find on:

\begintt
http://www-gap.dcs.st-and.ac.uk/gap
\endtt

and its mirrors at:

\begintt
http://www.math.rwth-aachen.de/~GAP
http://www.ccs.neu.edu/mirrors/GAP and
http://wwwmaths.anu.edu.au/research.groups/algebra/GAP/www/
\endtt

There you will find, amongst other things
\beginlist
\item{-} directions to the FTP sites from which you can download the
current {\GAP} distribution, any bugfixes, all accepted share packages,
and a selection of other contributions.
\item{-} the {\GAP} manual and an archive of the `gap-forum' mailing
list, formatted for reading with a Web browser, and indexed for
searching.
\item{-} information about {\GAP} developers, and about the email
addresses available for comment, discussion and support.
\item{-} advance information about and copies of presentations from
various {\GAP} workshops and events which take place from time to time
\endlist

I would particularly ask you to note five things:
\beginlist
\item{-} Any bugfixes which may have been made since this release.
\item{-} The  {\GAP} Forum -- an email discussion forum for comments,
discussions or questions about {\GAP}. You must subscribe to the list
before you can post to it, see the Web page for details.
\item{-} The email address `gap-trouble@dcs.st-and.ac.uk' to which you are asked to send
any questions or bug reports which do not seem likely to be of interest
to the whole {\GAP} Forum.
\item{-} The email address `gap@dcs.st-and.ac.uk' to which we ask you send a brief message 
when you install {\GAP}.
\item{-} The correct form of citation of {\GAP}, which we ask you use
whenever you publish scientific results obtained using {\GAP}.
\endlist

%%%%%%%%%%%%%%%%%%%%%%%%%%%%%%%%%%%%%%%%%%%%%%%%%%%%%%%%%%%%%%%%%%%%%%%%%
%%
%E  preface.tex . . . . . . . . . . . . . . . . . . . . . . . . ends here
